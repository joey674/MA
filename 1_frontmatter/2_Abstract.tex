%!TEX root = ../../main.tex

% Erste Seite hinter Titelseite
% Kurzfassung der Arbeit

\begin{center} {\LARGE Abstract} \end{center}

%Abstract hier
% present tense: Give background information; 1 general sentence about the socioeconomic or scientific relevance of your work
%present perfect: State a problem in 1 sentence with the conventional way of doing research in your particular discipline. Why is the conventional way of doing things in your field problematic? Expensive, complicated, imprecise?
%present tense: State your objective of your thesis: state the aim; This study presents / aims to
%past tense: State your methods (most important)
%past tense: State your most significant results (if you have enough space left discuss them briefly) 
%State your conclusion: take-home message for your readers; what do you want them to learn?

% [Background]
Osteoarthritis remains a global socioeconomic burden characterized by the irreversible degradation of Articular Cartilage (AC). Within the framework of the DWI Leibniz-Institute for Interactive Materials, research is directed toward developing fully automatic in vivo 3D bioprinting to overcome the limitations of traditional ex vivo scaffolding.
% [Problem]
Conventional dense 3D reconstruction techniques in endoscopic scenarios have struggled with weak structural textures, leading to insufficient robustness and precision. Furthermore, existing algorithmic frameworks have remained computationally heavy with prohibitive processing times, and their performance has been frequently compromised by dynamic surgical instruments or flowing tissues that obstruct the surgical field.
% [Objective]
This study aims to design a high-precision, real-time dense SLAM framework capable of maintaining robust 3D reconstruction even in the presence of intraoperative occlusions and dynamic disturbances.
% [Methods]
A Transformer-based feed-forward neural network was developed to perform high-fidelity dense 3D reconstruction. Besides a dedicated motion head was integrated and trained to identify and decouple dynamic surgical instruments and flowing tissues from the scenes. Furthermore, a real-time SLAM framework was constructed utilizing $SL(4)$ manifold representation. 
% [Results]
% The proposed system demonstrated a significant reduction in [Root Mean Square Error (RMSE)] for surface reconstruction while achieving a real-time processing speed of [XX] Frames Per Second (FPS). The motion masking head successfully improved reconstruction stability by  in occluded environments, and the SLAM trajectory maintained a low [Absolute Trajectory Error (ATE)].
/TODO: Add quantitative results/
\cleardoublepage
