%!TEX root = ./main.tex
\makenomenclature

% Überschrift
\renewcommand{\nomname}{List of Abbreviations and Symbols}

% Makro für Einheiten
\newcommand{\nomunit}[1]{\renewcommand{\nomentryend}{\hfill $\left[ #1 \right]$} }

% Nomenclaturabschnitte definieren
\renewcommand{\nomgroup}[1]{%
		\ifthenelse{\equal{#1}{S}}{\item[\textbf{List of Symbols}]}{%
				\ifthenelse{\equal{#1}{A}}{\item[\textbf{List of Abbreviations}]} {
					\ifthenelse{\equal{#1}{I}}{\item[\textbf{Index}]} {} } }    
				 }

% Einträge hinzufügen

% Symbols
% List the symbols used. These are examples only.
\nomenclature[S]{$\dot{m''}$}{specific mass flux \nomunit{\frac{kg}{s m^{2}}}}
\nomenclature[S]{$D$}{diffusion coefficient \nomunit{\frac{m^{2}}{s}}}
\nomenclature[S]{$C$}{mass concentration \nomunit{\frac{kg}{m^{3}}}}
\nomenclature[S]{...}{add symbol \nomunit{unit}}

% Indices
% List the indices used. Examples only.
\nomenclature[I]{$I$}{inside the wall directly at the gas side}%
\nomenclature[I]{$W$}{water}%
\nomenclature[I]{...}{add index}%

% Abbreviations
% List the abbreviations used. Examples only.
\nomenclature[A]{$TPMS$}{triply periodic minimal surface}%
\nomenclature[A]{$spez.$}{spec.}%
\nomenclature[A]{$i.A.$}{in general}%
\nomenclature[A]{...}{add abbreviations}

\cleardoublepage% or \clearpage
\markboth{\nomname}{\nomname}
\printnomenclature

