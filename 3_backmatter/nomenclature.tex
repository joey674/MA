%!TEX root = ./main.tex
\makenomenclature

% Überschrift
\renewcommand{\nomname}{Abkürzungs- und Symbolverzeichnis}

% Makro für Einheiten
\newcommand{\nomunit}[1]{\renewcommand{\nomentryend}{\hfill $\left[ #1 \right]$} }

% Nomenclaturabschnitte definieren
\renewcommand{\nomgroup}[1]{%
	\ifthenelse{\equal{#1}{S}}{\item[\textbf{Symbolverzeichnis}]}{%
		\ifthenelse{\equal{#1}{A}}{\item[\textbf{Abkürzungsverzeichnis}]} {
		  \ifthenelse{\equal{#1}{I}}{\item[\textbf{Indexverzeichnis}]} {} } }	
		 }

% Einträge hinzufügen

% Formelzeichen
% Hier die verwendeten Formel- und Symbolzeichen auflisten. Diese dienen nur als Beispiel
\nomenclature[S]{$\dot{m''}$}{spezifischer Massenstrom \nomunit{\frac{kg}{s m^{2}}}}
\nomenclature[S]{$D$}{Diffusionskoeffizient \nomunit{\frac{m^{2}}{s}}}
\nomenclature[S]{$C$}{Massenkonzentration \nomunit{\frac{kg}{m^{3}}}}
\nomenclature[S]{...}{Formelzeichen hinzufügen \nomunit{Einheit}}

% Indizes 
% Hier die verwendeten Indizes auflisten. Dies dient nur als Beispiel
\nomenclature[I]{$I$}{Innerhalb der Wand direkt an der Gasseite}%
\nomenclature[I]{$W$}{Wasser}%
\nomenclature[I]{...}{Index hinzufügen}%

% Abkürzungen
% Hier die verwendeten Abkürzungen auflisten. Dies dient nur als Beispiel
\nomenclature[A]{$TPMS$}{Dreifach periodische Minimaloberfläche}%
\nomenclature[A]{$spez.$}{spezifisch}%
\nomenclature[A]{$i.A.$}{im Allgemeinen}%
\nomenclature[A]{...}{Abkürzungen hinzufügen}

\cleardoublepage% or \clearpage
\markboth{\nomname}{\nomname}
\printnomenclature

