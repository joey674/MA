%!TEX program = xelatex
\documentclass[12pt]{avt-thesis}

\iffalse ANMERKUNGEN

- die Vorlage sollte mit XeLaTeX compiliert werden
- die Formatierung kann in avt-thesis.cls editiert werden
- für doppelseitiges oder einseitiges Drucken in avt-thesis.cls oneside bzw. twoside auskommentieren
- die Titelseite kann in pages/RWTH/Titelseite frei angepasst werden
- Referenzen im Bibtex Format können zb. über Google scholar gefunden und in literature.bib eingefügt werden

\fi
\graphicspath{{4_images/}}
\addbibresource{3_backmatter/literature.bib}

\begin{document}
% Titlepage, table of contents, etc.
    \frontmatter
    \pagenumbering{Roman}  
    \begin{titlepage}
        %!TEX root = ../../main.tex
\newgeometry{right=1.4cm,top=1cm,bottom=1cm,left=3cm}

\begingroup
\fontfamily{\sfdefault}\selectfont

\begin{flushright}
\includegraphics[width=11cm]{rwth_avt_rgb}
\end{flushright}

\begin{flushleft}
\normalsize{\begin{otherlanguage}{ngerman}\textbf{Diese Arbeit wurde vorgelegt am Lehrstuhl für Chemische Verfahrenstechnik}
\end{otherlanguage}\newline 
The present work was submitted to Chair of Chemical Process Engineering} 
\end{flushleft}
\vspace{2.5cm}
% Ende Instituskopf

% Beginn Titelseite
\begin{flushleft}

\begin{otherlanguage}{ngerman}
\Large{\textbf{Dichte 3D-Echtzeit-Rekonstruktion für die endoskopische Chirurgie mittels monokularer Sequenzen durch ein Transformer-basiertes Feed-Forward neuronales Netz}}\\
\end{otherlanguage}
\vspace{0.5cm}
\Large{\textbf{Real-Time Dense 3D Reconstruction for Endoscopic Surgery using Monocular Sequences via a Transformer-Based Feed-Forward Neural Network}}\\
\vspace{2.5cm}
\normalsize{Masterarbeit}\\
\normalsize{Master Thesis}\\
\vspace{2.5cm}
\normalsize{von / presented by}\\
\normalsize{Guan, Zhouyi (406259)}\\

\vspace{2.5cm}
\normalsize{Betreuer*in / Supervisor}\\
\normalsize{Yoo, Sang-Whon, M. Sc.}\\
\vspace{2.5cm}

\normalsize{Prof. Dr.-Ing. Matthias Wessling}\\
\end{flushleft}

\vfill

\begin{flushright}
\normalsize{Aachen, DATE}
\end{flushright}

\endgroup

\restoregeometry

\cleardoublepage


    \end{titlepage}
    %!TEX root = ../../main.tex

% Erste Seite hinter Titelseite
% Kurzfassung der Arbeit

\begin{center} {\LARGE Abstract} \end{center}

%Abstract hier
% present tense: Give background information; 1 general sentence about the socioeconomic or scientific relevance of your work
%present perfect: State a problem in 1 sentence with the conventional way of doing research in your particular discipline. Why is the conventional way of doing things in your field problematic? Expensive, complicated, imprecise?
%present tense: State your objective of your thesis: state the aim; This study presents / aims to
%past tense: State your methods (most important)
%past tense: State your most significant results (if you have enough space left discuss them briefly) 
%State your conclusion: take-home message for your readers; what do you want them to learn?

% [Background]
Osteoarthritis remains a global socioeconomic burden characterized by the irreversible degradation of Articular Cartilage (AC). Within the framework of the DWI Leibniz-Institute for Interactive Materials, research is directed toward developing fully automatic in vivo 3D bioprinting to overcome the limitations of traditional ex vivo scaffolding.
% [Problem]
Conventional dense 3D reconstruction techniques in endoscopic scenarios have struggled with weak structural textures, leading to insufficient robustness and precision. Furthermore, existing algorithmic frameworks have remained computationally heavy with prohibitive processing times, and their performance has been frequently compromised by dynamic surgical instruments or flowing tissues that obstruct the surgical field.
% [Objective]
This study aims to design a high-precision, real-time dense SLAM framework capable of maintaining robust 3D reconstruction even in the presence of intraoperative occlusions and dynamic disturbances.
% [Methods]
A Transformer-based feed-forward neural network was developed to perform high-fidelity dense 3D reconstruction. Besides a dedicated motion head was integrated and trained to identify and decouple dynamic surgical instruments and flowing tissues from the scenes. Furthermore, a real-time SLAM framework was constructed utilizing $SL(4)$ manifold representation. 
% [Results]
% The proposed system demonstrated a significant reduction in [Root Mean Square Error (RMSE)] for surface reconstruction while achieving a real-time processing speed of [XX] Frames Per Second (FPS). The motion masking head successfully improved reconstruction stability by  in occluded environments, and the SLAM trajectory maintained a low [Absolute Trajectory Error (ATE)].
/TODO: Add quantitative results/
\cleardoublepage

    %!TEX root = ../../main.tex

% Zweite Seite des Dokuments mit Aufgabenstellung

\begin{center}
\huge{Aufgabenstellung}
\end{center}
\normalsize 

\begin{otherlanguage}{ngerman}

%Aufgabenstellung hier

\end{otherlanguage}

\cleardoublepage

    
%    \includepdf[]{1_frontmatter/Versicherung.pdf}

\cleardoublepage
%   \includepdf[]{1_frontmatter/Versicherung.pdf}

\cleardoublepage %Falls es eine Gruppenarbeit ist
%   \includepdf[]{1_frontmatter/Versicherung.pdf}

\cleardoublepage

    %!TEX root = ../../main.tex


\begin{center}
\huge{Artificial Intelligence (AI) Statement}
\end{center}
\normalsize 

\begin{otherlanguage}{ngerman}

This thesis has been partially rewritten using an artificial intelligence (AI) model to improve readability. All ideas, considerations, and content are original and created by the author. Therefore, AI systems are not concerned with intellectual property. The AI usage rules stipulated by the Faculty of Mechanical Engineering of RWTH Aachen University on July 29, 2025, have been fully satisfied.

\end{otherlanguage}

\cleardoublepage
    
    \tableofcontents

% Thesis content - Add your chapters here
    \mainmatter
    \onehalfspacing
    
    % \include{0_templates/About_Latex}    
    % %authors: Arne Lüken and Lukas Griesberg 02/2022
%please send comments and additions to Arne Lüken arne.lueken@avt.rwth-aachen.de
\chapter{Wissenschaftliches Schreiben}

\section{Schreibstil}
    Ein wissenschaftlicher Text sollte in der Lage sein, komplexe Themen verständlich zu vermitteln. Sätze und Formulierungen sind hierfür möglichst kurz und klar zu gestalten. Vermeiden Sie unnötige Informationen und Wörter. Alle Wörter, die ausgelassen werden können, ohne den Sinn zu verändern, sollten ausgelassen werden. Der Sprachstil ist sachlich zu halten; ein Spannungsaufbau sollte nicht stattfinden. Verwenden Sie nach Möglichkeit Aktiv anstatt Passiv. Speziell im Englischen macht es den Text lebendiger:
    \begin{compactitem}
        \item \underline{Passiv:} "Der Katalysator ist auf der Nickelelektrode aufgebracht."
        \item \underline{Aktiv:} "Die Elektrode besteht aus einer mit Katalysator beschichteten Nickelplatte."
    \end{compactitem}
    Verwenden Sie das Präsens für allgemeine Informationen, Bezug auf Graphen oder Abbildungen, sowie bei der Interpretation von Ergebnissen und das Präteritum für das Zitieren von Forschungsergebnissen und bei Bezug auf eigene Messungen und Ergebnisse (z.B. "...wurde gemessen."). Achten Sie zudem auf eine einheitliche Wortwahl. Sprechen Sie beispielsweise nicht abwechselnd von "Mikrogelen", "Hydrogelen" oder "Partikeln", sondern beschränken Sie sich auf einen dieser Begriff.

\section{Struktur von Text und Graphen}

    Bevor mit dem Schreiben begonnen wird, sollte ein roter Faden erstellt werden. Zunächst sollten die Kapitel und Unterkapitel grob festgelegt werden, um einen Überblick über die Struktur der Arbeit zu erhalten. Falls Sie Unterkapitel verwenden, müssen es jeweils mindestens zwei Unterkapitel sein (z.B. 1.3.1. und 1.3.2.). Meist bietet es sich an, vor mehreren Unterkapiteln eine kurze Einleitung des Oberkapitels zu setzen. In der Regel sollten Unter-Unter-Kapitel (1.3.1.) ausreichend sein, um die Übersichtlichkeit der Arbeit zu gewährleisten.
    
    Bevor Sie zu schreiben anfangen, bietet es sich an, die Abbildungen und Graphen, die sie verwenden wollen, einzufügen und als Ausgangspunkt für Ihren Text zu nehmen. Sollten diese noch nicht fertig sein, kann es helfen, die erwarteten Ergebnisse als Skizze/Handzeichnung einzufügen und am Ende durch die fertigen Abbildungen zu ersetzen. Wichtig ist, dass jede verwendete Abbildung im Text referenziert wird. Ergebnisse, die Sie für wichtig erachten, jedoch nicht direkt ansprechen oder interpretieren, gehören in den Appendix. Alle relevanten Informationen sollten im Graphen oder der Bildunterschrift, welche bei Papern über mehrere Zeilen gehen darf, enthalten sein. Der Leser sollte ohne den Fließtext zu lesen, erkennen, was er auf der Abbildung sieht und was diese ausmacht, wie etwa relevante Versuchsbedingungen.
    
    Bei der Erstellung von Abbildungen und Graphen ist es empfehlenswert, Farbe nur falls unbedingt nötig zu verwenden. Es sollten keine Informationen verloren gehen, falls die Arbeit in schwarz-weiß gedruckt wird. Abbildungen sollten möglichst kontrastreich gestaltet werden. Falls farbige Abbildungen unvermeidbar sind, sollten Rot und Grün möglichst vermieden werden, da diese für farbenblinde Menschen problematisch sein können. Für Beschriftungen innerhalb von Abbildungen verwenden Sie bitte kurvige Linien (siehe Abbildung~\ref{fig:Beispielgraph}). Graphen sollten mit der Software \href{https://www.originlab.com/}{Origin} erstellt werden. Für Origin gibt es Vorlagen und eine \href{https://cvtwiki.avt.rwth-aachen.de/wiki/doku.php?id=sop:data_management:origin}{\textbf{\textit{Origin-Anleitung}}} im CVTwiki. Bilder und Schemata sollten immer einen Maßstabsbalken enthalten.
    
    \begin{figure}[h]	% Requires \usepackage{graphicx}
    		\begin{center}
    		  \includegraphics[width=\textwidth]{4_images/Beispiel-Graph.png}
    		  \caption[Beispielgraph]{Ein Beispielgraph mit Gestaltungshinweisen, die in \href{https://cvtwiki.avt.rwth-aachen.de/wiki/doku.php?id=sop:data_management:origin}{Origin} einfach und direkt angepasst werden können. Das Origintutorial kann im CVT Wiki unter https://cvtwiki.avt.rwth-aachen.de/wiki/doku.php?id=sop:data\_management:origin gefunden werden. Achtung, dieser Link funktioniert nicht im Code Editor wegen \_.}\label{fig:Beispielgraph}
    		\end{center}
    \end{figure}
    
    Um eine logische, nachvollziehbare Textstruktur zu erhalten, bietet sich die Verwendung von Absätzen an. Hierbei sollte jeder Absatz jeweils eine Idee behandeln und diese ausführen. In der Regel hat ein Absatz eine Länge von 5~-~10 Zeilen, so dass 3~-~5 Absätze auf eine Seite passen. Jeder Absatz sollte daher grob wie folgt strukturiert sein:
    \begin{compactitem}
        \item \underline{Topic Sentence:} Einleitender Satz, der die Grundaussage des Absatzes darlegt. Er beinhaltet sowohl das Thema (z.B. Mikrogele) als auch die Leitidee (z.B. anisometrisches Schwellen).
        \item \underline{Supporting Sentences:} Erklärung oder Unterstützung der Hauptidee durch Details in mehreren Sätzen.
        \item \underline{Concluding Sentence:} Fasst die wichtigsten Punkte zusammen und signalisiert das Ende des Absatzes. Er ist ähnlich strukturiert wie der Topic Sentence, beinhaltet aber mehr Informationen.
    \end{compactitem}
    
    
\FloatBarrier
\section{Gliederung}
Wissenschaftliche Arbeiten gliedern sich in der Regel in folgende Abschnitte:
\begin{enumerate}
    \item[] \textbf{Abstract} \vspace{-4 mm}
    \begin{compactitem}
        \item Sehr kurze Zusammenfassung der Thesis; Überblick für potenzielle Leser
        \item Inhalt:
        \begin{compactitem}
            \item Warum untersucht die Arbeit das Thema? (Motivation)
            \item Wie wurde das Thema behandelt? (Methoden)
            \item Was sind die Ergebnisse? (Ein bis zwei der wichtigsten Erkenntnisse, die am Ende herauskommen)
            %ONE general sentence
            %ONE sentence about the problem
            %state the objective
            %state the methods
            %state most significant results
            %take-home message
        \end{compactitem}
        \item Struktur:
        \begin{compactitem}
            \item Maximal 200 Wörter.
            \item Gegenwart, unpersönlicher Stil (vermeiden Sie \textit{ich} oder \textit{wir}).
        \end{compactitem}
    \end{compactitem}
    \item \textbf{Introduction / Motivation} \vspace{-4 mm}
    \begin{compactitem}
        \item Inhalt:
        \begin{compactitem}
            \item Warum wird das Thema untersucht? Was ist der Hintergrund der Arbeit? (Keine langen Ausschweifungen; kurz und prägnant Motivation erklären)
            \item Wer sind die Interessensgruppen? (z.B. für welche Anwendungen ist es interessant? Gibt es Produkte, die mithilfe der Ergebnisse entwickelt werden könnten? etc.)
            \item Um welche Problemstellung geht es genau und wie wird diese angegangen? 
            \item Was ist die Stuktur der Arbeit?
        \end{compactitem}
        \item Struktur:
        \begin{compactitem}
            \item Ca. eine Seite, höchstens zwei.
            \item Keine Unterkapitel.
        \end{compactitem}
    \end{compactitem}
    \item \textbf{Theoretical Background and State of the Art} \vspace{-4 mm}
    \begin{compactitem}
        \item Liefert Grundlage für die spätere Diskussion und Interpretation der Ergebnisse
        \item Inhalt:
        \begin{compactitem}
            \item Was sind die Grundlagen und Hintergründe des behandelten Themas?
            \item Wie ist der bisherige Stand der Forschung?
            \item Was wurde im behandelten Thema bereits am Institut getan?
            \item Was sind Schnittmengen zu anderen Forschungsfeldern? Wie werden die Methoden sonst angewendet?
        \end{compactitem}
        \item Struktur:
        \begin{compactitem}
            \item Unterkapitel ergeben Sinn.
            \item Es sollte eine visuelle Darstellung des State-of-the-art erstellt werden (z.B. chronologisch wie in Abbildung \ref{fig:ChronologicStateOfTheArt}).
        \end{compactitem}
    \end{compactitem}
    \item \textbf{Concept / Hypothesis} \vspace{-4 mm}
    \begin{compactitem}
        \item Inhalt:
        \begin{compactitem}
            \item Im Gegensatz zur Einleitung sollen hier aufbauend auf der wissenschaftlichen Lücke des State of the Arts die der Arbeit zugrundeliegenden Hypothesen detailliert dargestellt werden. Hier einige Beispiele für Hypothesen:
            \item Amin-imprägnierte GDEs erhöhen die lokale Konzentration von CO2 in der Umgebung des Katalysators und ermöglichen höhere partielle Stromdichten. Es exisitiert ein Optimum der Amin-Beladung, da bei zu hoher Beladung der Katalysator nicht mehr zugänglich ist.
            \item The mass transfer in a membrane contactor is dependent on the wetting behavior of the membrane
surface and the concentration polarization towards the membrane in the liquid phase. Thus, the mass transfer can be
optimized further by tailor-made surface properties of the membrane and turbulence promoters in the liquid phase

            \item Darauf aufbauend werden dann die spezifischen angewandten Methoden und die erwarteten Ergebnisse aufgelistet. Was? Warum?
        \end{compactitem}
        \item Struktur:
        \begin{compactitem}
            \item Circa eine Seite.
        \end{compactitem}
    \end{compactitem}
    \item \textbf{Materials and Methods} \vspace{-4 mm}
    \begin{compactitem}
        \item Genaue Beschreibung des verwendeten Versuchsaufbaus und -durchführung, damit die Versuche \textit{reproduzierbar} sind.
        \item Inhalt:
        \begin{compactitem}
            \item Kapitel sollte enthalten:
                 Materialien (inkl. Hersteller), Verwendete Software,
                 Mengen / Konzentrationen,
                 Versuchsaufbau,
                 Versuchsvorbereitung,
                 Kalibrierung,
                 Versuchsprotokolle
            \item \underline{Der Versuch sollte auf Basis dieses Kapitels replizierbar sein}
            \item Gehen Sie auf eventuell vorgenommene Veränderungen am Versuchsaufbau ein, die stattgefunden haben, und nennen Sie Probleme, die aufgetreten sind.
            \item \underline{Beachte:} Keine Ergebnisse und keine Bewertung in diesem Kapitel.
        \end{compactitem}
        \item Struktur:
        \begin{compactitem}
            \item Unterkapitel sind möglich.
            \item Tabellen können hilfreich sein.
            \item Versuchsaufbau als beschriftetes Foto / Skizze / Verfahrensfließbild.
            \item Kurz und prägnant.
        \end{compactitem}
    \end{compactitem}
    \item \textbf{Results and Discussion} \vspace{-4 mm}
    \begin{compactitem}
        \item Inhalt:
        \begin{compactitem}
            \item Alle relevanten Ergebnisse sollten aufgeführt werden und in den Kontext der Arbeit gebracht werden. Nicht für das Ergebnis der Arbeit relevante Ergebnisse sollten im Anhang untergebracht werden.
            \item Verwendung von Graphen und Schemata.
            \item Ergebnisse werden zunächst beschrieben (Results) und anschließend im wissenschaftlichen Kontext (Bezug zu State-of-the-art) analysiert und interpretiert (Discussion).
        \end{compactitem}
        \item Struktur:
        \begin{compactitem}
            \item Die Ergebnisse sollen nicht chronologisch, sondern aufeinander aufbauend dargestellt werden, wie man auch eine Geschichte erzählen würde.
            \item Der rote Faden sollte anhand von Abbildungen (Graphen) erkennbar sein.
            \item Results und Discussion können als getrennte Kapitel behandelt oder zusammengefasst werden.
        \end{compactitem}
    \end{compactitem}
    \item \textbf{Conclusion and Outlook} \vspace{-4 mm}
    \begin{compactitem}
        \item Zusammenfassung der zentralen Ergebnisse für jemanden, der die Arbeit nicht komplett gelesen hat.
        \item Inhalt:
        \begin{compactitem}
            \item Ergebnisse werden erneut aufgeführt und die in "Results and Discussion" erreichten Schlüsse werden wiederholt. 
            \item Es werden keine neuen Diskussionen / Punkte erörtert.
            \item \underline{Outlook:} Mögliche Ansätze für weiterführende Forschung. Vorschläge und Ideen, wie es weitergehen kann.
        \end{compactitem}
        \item Struktur:
        \begin{compactitem}
            \item 1-2 Seiten; je maximal eine Seite für Conclusion und Outlook.
            \item Kann je nach Bedarf in einzelne Kapitel für Conclusion und Outlook aufgeteilt werden.
        \end{compactitem}
    \end{compactitem}
\end{enumerate}

\begin{figure}[h!bt]	% Requires \usepackage{graphicx}
		\begin{center}
		  \includegraphics[width=1\textwidth]{4_images/Beispiel_StateArt.png}
		  \caption{Beispielhafte Visualisierung des State-of-the-art im Bereich "Magnetresonanztomographie (MRI) in Membranfiltration" aus den Jahren 2002 - 2018.}\label{fig:ChronologicStateOfTheArt}
		\end{center}
\end{figure}
 
    
     \chapter{Introduction}
\label{chap:introduction}

% \section{Background \& Socioeconomic Relevance}
Osteoarthritis (OA) stands as a premier cause of global disability, characterized by the irreversible degradation of Articular Cartilage (AC). Due to the avascular nature of AC, its intrinsic repair capacity is remarkably limited; even minor focal lesions often fail to heal, eventually leading to chronic pain and impaired mobility. While traditional ex vivo scaffolding techniques have been widely researched, recent advancements in medical robotics and materials science are paving the way for direct, in vivo 3D bioprinting. As envisioned within the major initiative at the DWI Leibniz-Institute for Interactive Materials, a fully automatic robotic process for in situ cartilage repair offers a transformative alternative by minimizing contamination risks and bypassing the time-consuming nature of conventional methods.\cite{Yoo2024}
% 骨关节炎(OA)是导致全球残疾的主要原因之一,其核心特征是关节软骨(AC)的不可逆退化。由于软骨组织缺乏血管,其自我修复能力极差,即使是微小的局灶性损伤也难以愈合,最终导致患者长期疼痛和行动受限。虽然传统的体外支架技术已被广泛研究,但医疗机器人和材料科学的进步正为“体内直接 3D 生物打印”铺平道路。正如 DWI 莱布尼茨互动材料研究所 的重大项目所愿,开发一种全自动机器人原位修复流程,不仅能降低污染风险,还能克服传统方法耗时长的弊端。

% \section{The Problem with Current Methods}
The success of in vivo bioprinting is fundamentally predicated on the precise 3D scanning and geometry estimation of the target lesion. Traditionally, pre-operative Magnetic Resonance Imaging (MRI) has been the gold standard; however, it suffers from systematic underestimation of lesion thickness and limited out-of-plane resolution. While photogrammetry presents a promising non-contact alternative, conventional dense 3D reconstruction algorithms often falter in the challenging endoscopic environment. These techniques are frequently hampered by weak textural features, prohibitive computational overhead, and the presence of dynamic occlusions---such as surgical instruments and flowing tissues---which compromise both the accuracy and the robustness required for real-time clinical intervention.\cite{Yoo2024}
% 体内生物打印的成功,根本上取决于对受损区域精准的 3D 扫描和几何估计。传统上,术前核磁共振成像(MRI)是金标准,但它存在系统性低估病变厚度的问题,且出平面的分辨率有限。虽然摄影测量(Photogrammetry)提供了一种非接触式的替代方案,但传统的稠密 3D 重建算法在挑战性的内窥镜环境下往往表现不佳。这些技术通常受限于微弱的纹理特征、过高的计算开销,以及手术器械和流动组织等动态遮挡物的干扰,难以满足实时临床手术所需的精度和鲁棒性。

% \section{Research Objective}
Building upon the foundational work at DWI, this thesis aims to bridge the gap between static photogrammetry and real-time surgical navigation. The primary objective is to develop a high-precision, real-time dense 3D reconstruction and SLAM framework. By leveraging Transformer-based architectures and advanced manifold optimization, this work seeks to provide a robust spatial mapping solution that can effectively handle intraoperative occlusions and provide the necessary geometric intelligence for autonomous robotic bioprinting.
% 承接 DWI 研究所的基础性工作,本论文旨在填补静态摄影测量与实时手术导航之间的技术空白。核心目标是开发一个高精度、实时的稠密 3D 重建与 SLAM 框架。通过利用 Transformer 架构和先进的流形优化算法,本研究力求提供一种稳健的空间建图方案,能够有效处理术中遮挡,并为全自动机器人生物打印提供必要的几何智能。

% \section{Proposed Methodology}
The proposed pipeline introduces a Transformer-based feed-forward neural network designed for high-fidelity geometry estimation, capturing global dependencies that traditional methods often miss. To ensure clinical reliability, a dedicated motion masking head is integrated to identify and decouple dynamic surgical disturbances from the static cartilage surface. Furthermore, the framework achieves high-efficiency real-time performance by utilizing an $SL(4)$ (Special Linear Group) manifold representation for SLAM, allowing for robust tracking under complex projective transformations inherent in monocular endoscopy.
% 所提出的流程引入了一种基于 Transformer 的前馈神经网络,旨在实现高保真度的几何估计,捕捉传统方法常常忽视的全局依赖关系。为了确保临床可靠性,集成了一个专门的运动掩码头,用于识别和分离动态手术干扰与静态软骨表面。此外,该框架通过利用 $SL(4)$(特殊线性群)流形表示实现了高效的实时性能,使其能够在单目内窥镜固有的复杂投影变换下实现稳健跟踪。

% \section{Thesis Structure}
The remainder of this thesis is structured as follows:

Section~\ref{sec:medical_background} discusses the medical background of knee AC and OA, emphasizing state-of-the-art treatment and 3D bioprinting techniques.

Section~\ref{sec:dense_3d_theory} introduces the theoretical foundations of dense 3D reconstruction technologies.

Section~\ref{sec:slam_theory} provides an overview of SLAM technologies and manifold optimization.

Section~\ref{sec:network_architecture} details the architectural design and training process of the proposed Transformer-based feed-forward neural network.

Section~\ref{sec:slam_framework} outlines the construction of the real-time dense SLAM framework built upon the aforementioned model.

Chapter~\ref{chap:results} analyzes the experimental results obtained from the pipeline, using well plates and simulated environments as precursors.

Chapter~\ref{chap:conclusion} concludes the thesis with a summary of findings and an outlook on potential optimizations for clinical application.
% 第 2.1 节:讨论膝关节软骨(AC)和骨关节炎(OA)的医学背景,重点介绍尖端的治疗方法和 3D 生物打印技术。
% 第 2.2 节:介绍稠密 3D 重建技术的理论基础。
% 第 2.3 节:概述 SLAM 技术及其流形优化理论。
% 第 3.1 节:详细阐述所提出的基于 Transformer 的前馈神经网络的架构设计与训练过程。
% 第 3.2 节:概述基于上述模型构建的实时稠密 SLAM 框架。
% 第 4 章:分析该流程在以孔板为前驱实验模型及模拟环境下的实验结果。
% 第 5 章:总结研究发现,并对未来临床应用的优化方向进行展望。
    \chapter{Theoretischer Hintergrund}
%\chapter{Theoretical Background} 
%\chapter{State of the Art}
    %\include{2_chapter/3_Hypothesis} 
    \chapter{Material und Methoden}
%\chapter{Materials and Methods}
    \chapter{Ergebnisse und Diskussion}
%\chapter{Results and Discussion}
    \chapter{Conclusion and Outlook}
\label{chap:conclusion}
%\chapter{Conclusion and Outlook}
    
% Nomenclature, bibliography, etc.
    %!TEX root = ./main.tex
\makenomenclature

% Überschrift
\renewcommand{\nomname}{Abkürzungs- und Symbolverzeichnis}

% Makro für Einheiten
\newcommand{\nomunit}[1]{\renewcommand{\nomentryend}{\hfill $\left[ #1 \right]$} }

% Nomenclaturabschnitte definieren
\renewcommand{\nomgroup}[1]{%
	\ifthenelse{\equal{#1}{S}}{\item[\textbf{Symbolverzeichnis}]}{%
		\ifthenelse{\equal{#1}{A}}{\item[\textbf{Abkürzungsverzeichnis}]} {
		  \ifthenelse{\equal{#1}{I}}{\item[\textbf{Indexverzeichnis}]} {} } }	
		 }

% Einträge hinzufügen

% Formelzeichen
% Hier die verwendeten Formel- und Symbolzeichen auflisten. Diese dienen nur als Beispiel
\nomenclature[S]{$\dot{m''}$}{spezifischer Massenstrom \nomunit{\frac{kg}{s m^{2}}}}
\nomenclature[S]{$D$}{Diffusionskoeffizient \nomunit{\frac{m^{2}}{s}}}
\nomenclature[S]{$C$}{Massenkonzentration \nomunit{\frac{kg}{m^{3}}}}
\nomenclature[S]{...}{Formelzeichen hinzufügen \nomunit{Einheit}}

% Indizes 
% Hier die verwendeten Indizes auflisten. Dies dient nur als Beispiel
\nomenclature[I]{$I$}{Innerhalb der Wand direkt an der Gasseite}%
\nomenclature[I]{$W$}{Wasser}%
\nomenclature[I]{...}{Index hinzufügen}%

% Abkürzungen
% Hier die verwendeten Abkürzungen auflisten. Dies dient nur als Beispiel
\nomenclature[A]{$TPMS$}{Dreifach periodische Minimaloberfläche}%
\nomenclature[A]{$spez.$}{spezifisch}%
\nomenclature[A]{$i.A.$}{im Allgemeinen}%
\nomenclature[A]{...}{Abkürzungen hinzufügen}

\cleardoublepage% or \clearpage
\markboth{\nomname}{\nomname}
\printnomenclature



    \nocite{Yoo2024}
    \printbibliography[heading=bibintoc]
    \listoffigures
    \listoftables
    \appendix
    \include{3_backmatter/Appendix}
    
    
\end{document}
